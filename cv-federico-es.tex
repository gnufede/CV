\documentclass[margin,line]{resume}
\usepackage[utf8]{inputenc} 
%\usepackage[spanish]{babel} 

\oddsidemargin -.5in
\evensidemargin -.5in
\textwidth=6.0in
\itemsep=0in
\parsep=0in

\begin{document}

\name{Federico G. Mon Trotti \vspace*{.1in}}

\begin{resume}
\section{\sc Información de Contacto}
\vspace{.05in}
\begin{tabular}{@{}p{0.5in}p{2in}}
{\it Teléfono:} &(+34) 618 128 435 \\
{\it E-mail:} &{ federico.mon@gmail.com}\\
{\it LinkedIn:} &{ \tt http://es.linkedin.com/in/federicomon}\\
\end{tabular}

\section{\sc Experiencia Profesional}
{\bf Universidad Complutense de Madrid}, Madrid, España

\vspace{-.3cm}
{\em Responsable de la Oficina de Tecnologías Abiertas y Software Libre} \hfill {\bf Enero 2011 - Actualidad
}

\vspace{-.3cm}
{\em Becario en el Departamento de Ingeniería del Software e Inteligencia Artificial asociado al proyecto MILES} \hfill {\bf Octubre 2010 -
Diciembre 2010}


{\bf Sinpro Web Solutions}, Eindhoven, Países Bajos

\vspace{-.3cm}
{\em Desarrollador web y administrador de sistemas} \hfill {\bf Octubre 2009 -
Enero 2010}

\section{\sc Educación}
{\bf Universidad Complutense de Madrid}, Madrid, España\\
\vspace*{-.1in}
\begin{list1}
\item[] Ingeniería en Informática (Junio 2011)
\begin{list2}
\vspace*{.05in}
\item Projecto Fin de Carrera:  ``Inteligencia Ambiental en Dispositivos Móviles'' 
\item Directores:  Manuel Prieto, Marco A. Gómez-Martín
\end{list2}
\end{list1}


{\bf Eindhoven University of Technology}, Eindhoven, Países Bajos\\
\vspace*{-.1in}
\begin{list1}
\item[] Programa Erasmus (Septiembre 2009 - Julio 2010)
\end{list1}

\section{\sc Más Formación } 
\begin{list1}
\item[] {\bf Curso "Desarrollo para móviles Open Source"} Impartido por Elondra
\item[] {\bf Inglés:} Acreditado como nivel B2 del marco común europeo de referencia
\item[] {\bf Permiso de conducción:} B
\end{list1}

\section{\sc Proyectos de Software } 
\begin{list1}
\item[] {\bf HD Lorean}
Solución de copias de seguridad en tiempo real para GNU/Linux.\\
Tecnologías usadas: GTK, Python, Xdelta3, Inotify, Sqlite.\\

\item[] {\bf Context Manager}
Aplicación para la plataforma Android, que hace reaccionar al teléfono 
de acuerdo con su contexto (geolocalización y hora), dependiendo de las reglas definidas por el usuario.\\
Tecnologías usadas: Android SDK, APIs de Geolocalización, Base de datos,
Contactos, Modos de Teléfono.
\end{list1}

\section{\sc Tecnologías} 
\begin{list1}
\item[]{\bf Languajes:} C, C++, Java (J2SE, J2ME, Android), Python,
Javascript, HTML, SQL, Unix shell scripts, Prolog, Haskell, Pascal, VHDL,
Ensamblador (m68k).
\item[]{\bf Sistemas de Control de Revisiones:} Git, Mercurial, Subversion, Bazaar.
\item[]{\bf Sistemas Operativos:} Unix/Linux.
\end{list1}
%\begin{list2}
%\item {\bf Languajes:} C, C++, Java (J2SE, J2ME, Android), Python,
%Javascript, HTML, SQL, Unix shell scripts, Prolog, Haskell, Pascal, VHDL,
%Ensamblador (m68k).
%\item {\bf Sistemas de Control de Revisiones:} Git, Subversion, Bazaar.
%\item {\bf Sistemas Operativos:} Unix/Linux. 
%\end{list2}

\section{\sc Experiencia Personal}
{\bf Grupo de Usuarios de Linux de la Universidad Complutense de
Madrid}, Madrid, España

\vspace{-.3cm}
{\em Socio Fundador, Presidente} \hfill {\bf Marzo 2011 - Actualidad}


%{\bf Delegación de Alumnos, Facultad de Informática, Universidad Complutense de
%Madrid}, Madrid, España\\

%\vspace{-.3cm}
%{\em Representante de Estudiantes} \hfill {\bf Abril 2004 - Abril 2008}\\
%{\em Presidente} \hfill {\bf Abril 2006 - Abril 2008}

%{\bf RITSI}, España

%\vspace{-.3cm}
%{\em Vocal} \hfill {\bf  Noviembre 2005 - Abril 2006}\\
%{\em Presidente} \hfill {\bf  April 2006 - Abril 2007}

%{\bf ATI}, España
%
%\vspace{-.3cm}
%{\em 2º Vocal} \hfill {\bf %%EN MADRID
%  Nov 2006 - Dic 2009}\\

\end{resume}
\end{document}

