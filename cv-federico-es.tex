\documentclass[margin,line]{resume}
\usepackage[utf8]{inputenc} 
%\usepackage[spanish]{babel} 

\oddsidemargin -.5in
\evensidemargin -.5in
\textwidth=6.0in
\itemsep=0in
\parsep=0in

\begin{document}

\name{Federico G. Mon Trotti \vspace*{.1in}}
\address{Ingeniero en Informática}

\begin{resume}
\section{\sc Información de Contacto}
\vspace{.05in}
\begin{tabular}{@{}p{0.5in}p{2in}}
{\it Teléfono:} &(+34) 618 128 435 \\
{\it E-mail:} &{ federico.mon@gmail.com}\\
{\it LinkedIn:} &{ \tt http://es.linkedin.com/in/federicomon}\\
\end{tabular}

\section{\sc Experiencia Profesional}
{\bf Radmas Technologies}, Madrid, España

\vspace{-.3cm}
{\bf \em Web Developer, Backend y Frontend} \hfill
{ Noviembre 2011 - Febrero 2012
}

\begin{list2}
\vspace*{.05in}
\item {\bf Funciones: }{Desarrollo web, implantación, pruebas.}
\item {\bf Tecnologías utilizadas: }{PHP (Symfony), MySQL, HTML5, CSS,
Javascript, JQuery, AJAX, GIT, Apache, Sistemas GNU/Linux de alta
disponibilidad, VMWare ESXi, Heartbeat, PeaceMaker.}
\end{list2}

{\bf Universidad Complutense de Madrid}, Madrid, España

\vspace{-.3cm}
{\bf \em Becario en la Oficina de Tecnologías Abiertas y Software Libre} \hfill
{ Enero - Octubre 2011
}
\begin{list2}
\vspace*{.05in}
\item {\bf Funciones: }{Administración de sistemas (servidores y ordenadores
personales), docencia, divulgación.}
\item {\bf Tecnologías utilizadas: }{SSH, LDAP, NFS, Rsync, Nagios, PHP
(Drupal), MySQL, Apache, CentOS GNU/Linux, Ubuntu GNU/Linux, Solaris 10.}
\end{list2}


%%\vspace{-.3cm}
%%{\bf \em Becario en el Depto. de Ingeniería del Software e I.A.} \hfill { Octubre -
%%Diciembre 2010}
%%\begin{list2}
%%\vspace*{.05in}
%%\item {\bf Funciones: }{Desarrollo de software.}
%%\item {\bf Tecnologías utilizadas: }{Python, SQLite, sockets.}
%%\end{list2}

{\bf Sinpro Web Solutions}, Eindhoven, Países Bajos

\vspace{-.3cm}
{\bf \em Desarrollador web y administrador de sistemas} \hfill { Octubre 2009 -
Enero 2010}
\begin{list2}
\vspace*{.05in}
\item {\bf Funciones: }{Desarrollo web, diseño web, implantación, administración de
sistemas.}
\item {\bf Tecnologías utilizadas: }{Gimp, HTML, CSS, PHP (Magento), MySQL,
Apache, Ubuntu GNU/Linux.}
\end{list2}

\section{\sc Educación}
{\bf Universidad Complutense de Madrid}, Madrid, España\\
\vspace*{-.1in}
\begin{list1}
\item[] {\bf \em Ingeniería en Informática }(Junio 2011) 
\begin{list2}
\vspace*{.05in}
\item Projecto Fin de Carrera:  ``Inteligencia Ambiental en Dispositivos Móviles'' 
\item Directores:  Manuel Prieto, Marco A. Gómez-Martín
\end{list2}
\end{list1}


{\bf Eindhoven University of Technology}, Eindhoven, Países Bajos\\
\vspace*{-.1in}
\begin{list1}
\item[] {\bf \em Programa Erasmus} (Septiembre 2009 - Julio 2010) 
\end{list1}

\section{\sc Proyectos de Software } 
\begin{list1}
\item[] {\bf HD Lorean}
Solución de copias de seguridad en tiempo real para GNU/Linux.
\vspace*{.05in}
\begin{list2}
\item {\bf Tecnologías utilizadas:} GTK, Python, Xdelta3, Inotify, Sqlite.\\
\end{list2} 

\item[] {\bf Context Manager}
Aplicación para la plataforma Android, que hace reaccionar al teléfono de acuerdo con su contexto (geolocalización y hora), dependiendo de las reglas definidas por el usuario.
\vspace*{-.13in}
\begin{list2}
\item {\bf Tecnologías utilizadas:} Android SDK, APIs de Geolocalización, Base de datos,
Contactos, Modos de Teléfono.
\end{list2}
\end{list1}

%%\newpage
\section{\sc Tecnologías} 
\begin{list1}
\item[]{\bf Preferidas:} Python, Javascript, HTML, CSS, Java (J2SE, J2ME, Android). 
\item[]{\bf Otros Lenguajes y Frameworks:} Symfony, PHP, JQuery, C, C++, SQL, Bash, Prolog, Haskell, Pascal, VHDL, Ensamblador (m68k).
\item[]{\bf Sistemas de Control de Revisiones:} Git, Mercurial, Subversion, Bazaar.
\item[]{\bf Administración de Sistemas :} Unix/Linux, Solaris.
\end{list1}
%\begin{list2}
%\item {\bf Languajes:} C, C++, Java (J2SE, J2ME, Android), Python,
%Javascript, HTML, SQL, Unix shell scripts, Prolog, Haskell, Pascal, VHDL,
%Ensamblador (m68k).
%\item {\bf Sistemas de Control de Revisiones:} Git, Subversion, Bazaar.
%\item {\bf Sistemas Operativos:} Unix/Linux. 
%\end{list2}

\section{\sc Más Formación } 
\begin{list1}
\item[] {\bf Curso "Desarrollo para móviles Open Source"} Impartido por Elondra
\item[] {\bf Seminario "Habilidades Directivas"} Impartido por Accenture
\item[] {\bf Inglés:} Acreditado como nivel B2 del marco común europeo de referencia
\item[] {\bf Permiso de conducción:} B
\end{list1}


\section{\sc Experiencia Personal}
{\bf Grupo de Usuarios de Linux de la Universidad Complutense de
Madrid}, Madrid, España

\vspace{-.3cm}
{\bf \em Socio Fundador, Presidente} \hfill { Marzo 2011 - Actualidad}


{\bf Delegación de Alumnos, Facultad de Informática, Universidad Complutense de
Madrid}, Madrid, España\\

\vspace{-.7cm}
{\bf \em Representante de Estudiantes} \hfill {Abril 2004 - Abril 2008}\\
{\bf \em Presidente} \hfill { Abril 2006 - Abril 2008}

{\bf RITSI}, España

\vspace{-.3cm}
{\bf \em Vocal} \hfill { Noviembre 2005 - Abril 2006}\\
{\bf \em Presidente} \hfill { April 2006 - Abril 2007}

{\bf ATI}, España

\vspace{-.3cm}
{\bf \em 2º Vocal en el Capítulo de Madrid} \hfill {%%EN MADRID
  Nov 2006 - Dic 2009}\\

\end{resume}
\end{document}

